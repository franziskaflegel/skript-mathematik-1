\chapter{Zahlenbereiche}
\section{Natürliche Zahlen}
Die natürlichen Zahlen 1, 2, 3, 4, \ldots ~sind die ältesten Zahlen, die sich aus dem Zählen von Gegenständen entwickelt haben.
Dabei nimmt man an, dass man immer weiterzählen kann.\\

\begin{defn}{Natürliche Zahlen}
	Die Menge der natürlichen Zahlen wird mit $\N$ bezeichnet.
	Man schreibt: $\N = \{ 1, 2, 3, 4, \ldots \}$
\end{defn}

\paragraph{Achtung!}
Die Zahl 0 wird dabei {\bfseries nicht} mit dazu genommen.
Möchte man die Null mit einschliessen, so verwenden wir die Schreibweise $\N_0$.\\

\begin{defn}{Natürliche Zahlen zusammen mit Null}
	Die Menge der natürlichen Zahlen wird mit $\N_0$ bezeichnet.
	Man schreibt: $\N_0 = \{ 0, 1, 2, 3, 4, \ldots \}$
\end{defn}

\paragraph{Zweck.}
Die natürlichen Zahlen braucht man z.B. um
\begin{itemize}
	\item Anzahlen anzugeben oder
	\item Rangplätze oder Reihenfolgen festzulegen.
\end{itemize}

\begin{example}
	Wir haben heute 6 Lektionen. (Anzahl)
\end{example}
\begin{example}
	Heute ist der erste Tag des Monats. (Rangplatz)
\end{example}

\section{Ganze Zahlen}
\section{Rationale Zahlen}
\section{Kann man alle Zahlen als Bruch darstellen?}
% Hier muss noch etwas Geschichte hin.

